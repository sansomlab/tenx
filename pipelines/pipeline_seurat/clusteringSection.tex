\subsection{Assesment of the tSNE perplexity hyperparameter}

\input{\clusterDir/tSNE.perplexity}
\clearpage

\subsection{tSNE parameters}

The parameters chosen for the subsequent tSNE plots are:

\begin{itemize}
\item Perplexity: \tSNEPerplexity
\item Maximum number of iterations: \tSNEMaxIter
\item Use fast approximation: \tSNEFast
\end{itemize}

\subsection{tSNE plots colored by factors of interest (e.g. by cluster)}

\input{\clusterDir/plot.rdims.tsne.factor}

\subsection{UMAP plots colored by factors of interest}

\input{\clusterDir/plot.rdims.umap.factor}

\subsection{Breakdown of cell numbers by factors of interest (e.g. by cluster)}

\input{\clusterDir/number.plots}

\subsection{Exploring the relationship between the clusters}

The distances between the clusters was assessed using the ``BuildClusterTree'' function in the Seurat package, which ``constructs a phylogenetic tree relating the ``average'' cell from each identity class''.

\begin{figure}[H]
\includegraphics[width=1.0\textwidth,height=0.9\textheight,keepaspectratio]{{{\clusterDir/cluster.dendrogram}}}
\caption{Visualisation of inter-cluster distances (gene-based)}
\end{figure}
